\documentclass[a4paper,11pt]{article}

\usepackage[italian]{babel}
\usepackage{soul}
\usepackage{mathtools}
\usepackage{amssymb,amsmath,amsfonts}
\usepackage[utf8]{inputenc}
\usepackage{graphicx}
\usepackage{geometry}
\usepackage{float}
\usepackage{csquotes}
\usepackage{hyperref}
\usepackage{fancyhdr}
\usepackage{gensymb}
\usepackage{units}
\usepackage{hhline}
\usepackage{color}
\usepackage[export]{adjustbox}
\usepackage[nottoc,numbib]{tocbibind}
\usepackage[super,comma]{natbib}
\usepackage{titling}
\usepackage{datetime}

%\renewcommand{\today}{\thisdayofweekname\ \theday\ \monthname\ \the\year}

\geometry{a4paper, left=30mm, right=30mm, top=30mm, bottom=30mm}
\definecolor{pantone294}{cmyk}{1,0.6,0,0.2}

\title{XXX NOME DELL'ESPERIMENTO XXX} %modifica togliendo le xxx
\author{XXX NOMI STUDENTI XXX}
\date{\today}

\pagestyle{fancy}
\lfoot{Università di Pisa}
\rfoot{xxx nome esperimento xxx} 

\begin{document}
	\newgeometry{left=14mm, right=13.5mm, top=13.5mm, bottom=30mm}
	\begin{titlepage}
		\thispagestyle{empty}
		\begin{figure}
			\includegraphics[width=31.5mm,right]{./Cherubino}
		\end{figure}
		\vspace*{-43mm}\hspace{-6mm}\textbf{\textcolor{pantone294}{\large{Dipartimento di Fisica}}}\\\\\\\\\\
		
		\vspace{30mm}
		\begin{center}
			\textcolor{pantone294}{\huge{xxx nome del corso  xxx}}\\\vspace*{7mm} %esempio: Laboratorio I, II o III
			\textcolor{pantone294}{\huge{\textbf{\thetitle}}}\\\vspace*{10mm}
			\textcolor{pantone294}{\theauthor}\\\vspace*{10mm}
			\textcolor{pantone294}{\thedate}\\\vspace*{20mm}
			\begin{tabular}{ll}
				\textbf{Studenti:} & xxx nome xxx (xxx numero di matricola xxx) \\
				& xxx mail xxx\\ \\
				& xxx nome xxx (xxx numero di matricola xxx) \\
				& xxx email xxx \\ \\
				\textbf{Professore:} & xxx nome del prof. xxx \\ \\
				
			\end{tabular}
		\end{center}
	\end{titlepage}
	\makeatother
	\restoregeometry
	\newpage
	
	\tableofcontents
	
	\vspace{2cm}
	\section{Abstract}
	%qui fai una descrizione breve e coincisa dell'esperimento, ci metti le nozioni base di fisica e ci scrivi i risultati più importanti ottenuti
	\newpage
	
	
	
	% i prossimi capitoli sono come di solito organizzo una relazione io. te la puoi organizzare anche diversamente, se vuoi.
	\section{Descrizione dell'esperimento}
% qua descrivi gli strumenti di misura e come hai preso i dati. ricavati le formule che usi in questa sezione. non mettere i numeri, vengono dopo
	\section{Analisi dei dati}
%qua ti calcoli cosa ti devi calcolare. alla fine di questa sezione ti trovi il risultato e la sua incertezza. è in questa sezione che ci metti grafici, fit, calcolo dell'errore etc.
%consiglio: metti i dati nelle tabelle alla fine e quando ti riferisci a determinati dati scrivi: utilizzando i dati in Tabella 1...        è più elegante.

	%a titolo di esempio, questo è il comando per mettere pdf, jpeg, png etc.
			%\begin{figure}[H]
			%	\centering
			%	\includegraphics[width=13cm]{measured_data.pdf}
			%	\caption{xxx didascalia figura xxx}
			%\end{figure}
			
	\section{Commenti conclusivi}
%qua ci scrivi se i risultati del tuo esperimento coincidono con i valori dei tuoi riferimenti bibliografici. quando lo paragoni a qualcosa mettici sempre la fonte, tipo il libro (autore: titolo, editore, luogo, anno) o il sito (mettici ora e data di quando lo controlli con una nota a piè di pagina)
	%in più in questa sezione ci devi scrivere cosa ti può essere andato male, tipo quali sono le cause per cui non ottieni i risultati previsti. 
	
	
	\section{Allegati}

\subsection{Tabelle}
%ne metto una per fare un esempio
%qua se vuoi puoi mettere i dati presi in laboratorio
\begin{table}[ht] \centering \caption {\bf Längenänderung}	
\begin{tabular}{c|c|c|c|}
Masse in $g$ & 1. gemesste Länge in $mm$ & 2. gemesste Länge in $mm$ & Mittlwert $\overline l$ in $mm$ \\ \hline
0 & 10,36 & 10,37 & 10,365\\
50 & 10,50 & 10,51 & 10,505\\
100 & 10,63 & 10,65 & 10,640\\
200 & 10,89 & 10,915 & 10, 9025\\
300 & 11,14 & 11,17 & 10,155\\
400 & 11,40 & 11,42 & 10,41\\
500 & 11,62 & 11,64 & 11,63\\
600 & 11,85 & 11,88 & 11,865\\
700 & 12,10 & 12,11 & 12,105\\
800 & 12,34 & 12,35 & 12,345
\end{tabular}
\end{table}
	

%last but not least la bibliografia
	\begin{thebibliography}{1}
		%\bibitem{label}AUTORE, \textit{titolo}, editore, luogo, anno
		%metto sempre una per fare un esempio
		\bibitem{PraktikumsScriptMechanik}Dr. Uwe Müller: \textit{Physikalisches Grundpraktikum: Mechanik und Thermodynamik}, Berlin, 2012
	\end{thebibliography}
\end{document}
